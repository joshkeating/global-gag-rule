\documentclass[11pt,]{article}
\usepackage[left=1in,top=1in,right=1in,bottom=1in]{geometry}
\newcommand*{\authorfont}{\fontfamily{phv}\selectfont}
\usepackage[]{mathpazo}


  \usepackage[T1]{fontenc}
  \usepackage[utf8]{inputenc}



\usepackage{abstract}
\renewcommand{\abstractname}{}    % clear the title
\renewcommand{\absnamepos}{empty} % originally center

\renewenvironment{abstract}
 {{%
    \setlength{\leftmargin}{0mm}
    \setlength{\rightmargin}{\leftmargin}%
  }%
  \relax}
 {\endlist}

\makeatletter
\def\@maketitle{%
  \newpage
%  \null
%  \vskip 2em%
%  \begin{center}%
  \let \footnote \thanks
    {\fontsize{18}{20}\selectfont\raggedright  \setlength{\parindent}{0pt} \@title \par}%
}
%\fi
\makeatother




\setcounter{secnumdepth}{0}


\usepackage{graphicx}
% We will generate all images so they have a width \maxwidth. This means
% that they will get their normal width if they fit onto the page, but
% are scaled down if they would overflow the margins.
\makeatletter
\def\maxwidth{\ifdim\Gin@nat@width>\linewidth\linewidth
\else\Gin@nat@width\fi}
\makeatother
\let\Oldincludegraphics\includegraphics
\renewcommand{\includegraphics}[1]{\Oldincludegraphics[width=\maxwidth]{#1}}

\title{Global Effects Of The Mexico City Policy  }



\author{\Large Joshua Keating, Peter Freschi, and John Harrison\vspace{0.05in} \newline\normalsize\emph{University of Washington}  }


\date{}

\usepackage{titlesec}

\titleformat*{\section}{\normalsize\bfseries}
\titleformat*{\subsection}{\normalsize\itshape}
\titleformat*{\subsubsection}{\normalsize\itshape}
\titleformat*{\paragraph}{\normalsize\itshape}
\titleformat*{\subparagraph}{\normalsize\itshape}


\usepackage{natbib}
\bibliographystyle{apsr}



\newtheorem{hypothesis}{Hypothesis}
\usepackage{setspace}

\makeatletter
\@ifpackageloaded{hyperref}{}{%
\ifxetex
  \usepackage[setpagesize=false, % page size defined by xetex
              unicode=false, % unicode breaks when used with xetex
              xetex]{hyperref}
\else
  \usepackage[unicode=true]{hyperref}
\fi
}
\@ifpackageloaded{color}{
    \PassOptionsToPackage{usenames,dvipsnames}{color}
}{%
    \usepackage[usenames,dvipsnames]{color}
}
\makeatother
\hypersetup{breaklinks=true,
            bookmarks=true,
            pdfauthor={Joshua Keating, Peter Freschi, and John Harrison (University of Washington)},
             pdfkeywords = {Mexico City policy, Global Gag Rule, NGOs, family planning, maternal
mortality},  
            pdftitle={Global Effects Of The Mexico City Policy},
            colorlinks=true,
            citecolor=blue,
            urlcolor=blue,
            linkcolor=magenta,
            pdfborder={0 0 0}}
\urlstyle{same}  % don't use monospace font for urls



\begin{document}
	
% \pagenumbering{arabic}% resets `page` counter to 1 
%
% \maketitle

{% \usefont{T1}{pnc}{m}{n}
\setlength{\parindent}{0pt}
\thispagestyle{plain}
{\fontsize{18}{20}\selectfont\raggedright 
\maketitle  % title \par  

}

{
   \vskip 13.5pt\relax \normalsize\fontsize{11}{12} 
\textbf{\authorfont Joshua Keating, Peter Freschi, and John Harrison} \hskip 15pt \emph{\small University of Washington}   

}

}







\begin{abstract}

    \hbox{\vrule height .2pt width 39.14pc}

    \vskip 8.5pt % \small 

\noindent The recent restoration of the Mexico City policy has brought the issue
of U.S. federal funding for non-governmental organisations (NGOs) to the
forefront of the public consciousness. The conflict lies in the policy's
mandate that NGOs that receive federal funds agree to neither perform
nor actively promote abortion as a method of family planning. This means
that any organization that wishes to continue to receive US funding must
cease to provide abortion related services such as; education, family
planning, counseling, and training if they wish to keep their federal
funding. This has the potential to disrupt important health care
services, including maternal health care, in regions of the world where
additional funding is most needed. In our research we looked at some of
the current and historical levels of contraceptive use, family planning
methods in specific regions, maternal mortality and induced abortion
rates, and the flow of federal funds into some of the largest NGOs.


\vskip 8.5pt \noindent \emph{Keywords}: Mexico City policy, Global Gag Rule, NGOs, family planning, maternal
mortality \par

    \hbox{\vrule height .2pt width 39.14pc}



\end{abstract}


\vskip 6.5pt

\noindent  \section{Introduction}\label{introduction}

The issue of abortion in the United States is one that is fiercely
debated and divides large segments of the population. These divides are
inherently partisan and therefore have led to legislation. In 1984 the
Reagan administration signed into law the ``Mexico City policy'', named
for the city in which it was signed. The policy requires all non
governmental organizations operating abroad to refrain from performing,
advising on or endorsing abortion as a method of family planning if they
wish to continue to receive receive federal funding.\footnote{The policy
  mandates that NGOs pledge to not ``perform or actively promote
  abortion as a method of family planning'' with non-U.S. funds as a
  condition for receiving U.S. global family planning assistance.} The
policy has had a tumultuous history, either being reinstated or
rescinded depending on the party affiliation of the current president.
Starting with Bill Clinton in 1993 the policy was rescinded, then
restored by George W. Bush in 2001, rescinded again by Barack Obama in
2009, and most recently restored by Donald J. Trump in 2017\footnote{As
  of Jan. 23, 2017, the policy extends to any other U.S. global health
  assistance, including U.S. global HIV and maternal and child health.}.

At its heart, the ban is propelled by the desire to limit the use of
U.S. taxpayer dollars to pay for abortion or abortion-related services.
The consequences of this ban include the termination of abortion related
services such as; education, family planning, counseling, and training.
The all or nothing nature of the ban would also cut off funds for non
abortion related health services that would be offered at non
governmental health providers. This presents some serious public health
concerns for areas of the world where these services are essential for
maternal health and prenatal care.

We hypothesized that the institution of the ban could lead to a
reduction in family planning services which could then lead to an
increase in induced abortions. By restricting the flow of funds into
health clinics that provide services related to abortion among others,
the U.S. government restricts the amount of care that can be provided to
those utilizing those services. The aim of this project was to gather,
analyze, and present data on the past effects of the Global Gag Rule and
create a resource that assists our audience in understanding the scope
of the policy, its observable impact on women's health, and the pathways
through which it influences public health policy decisions outside of
the United States.

\section{Related Work}\label{related-work}

The United Nations' Department of Economic and Social Affairs has
conducted various research via nationally-representative surveys. Their
dataset\footnote{World Contraceptive Use 2016, includes country-specific
  estimates of these and other indicators, based on survey data
  available as of April 2016
  \href{http://www.un.org/en/development/desa/population/publications/dataset/contraception/wcu2016.shtml}{dataset
  link}} (World Contraceptive Use 2016) contains information about
prevalence and unmet family planning needs for 195 different
countries/areas of the world from 1950 to 2015. This information is
broken down by type of contraceptive and specific demand for family
planning through different methods, which was gathered through about 15
conducted surveys across the world. These types of indicators are
effective in assessing the progress of universal reproductive
health-care and family planning information in a geographical area. The
global gag rule has been known to threaten the funding of certain
health-care organizations, which may have effects on the prevalence of
contraceptive and family planning methods.

The World Health Organization (WHO) published a study\footnote{United
  States aid policy and induced abortion in sub-Saharan Africa
  \href{http://www.who.int/bulletin/volumes/89/12/11-091660/en/}{study},
  \href{http://www.who.int/bulletin/volumes/89/12/BLT-11-091660-table-T1.html}{dataset}}
in partnership with the Stanford University Department of Medicine in
2011 analysing the effects of the ``Mexico City Policy'' on a set of
sub-Saharan African countries. Specifically, this study attempted to
determine whether a relationship exists between the reinstatement of the
Mexico City Policy and the probability that a sub-Saharan African woman
will have an induced abortion. The Stanford researchers looked at the
relationship between a country's exposure to the Mexico City Policy and
the odds of abortion among women of reproductive age between 1994 and
2008. In this way they were able to find a statistically
significant\footnote{Women living in highly exposed countries had 2.73
  (95\% CI: 1.95--3.82) times the odds of having an induced abortion
  after the policy's reinstatement than during the period from 1994 to
  2000 or than women living in less exposed countries.
  \href{http://www.who.int/bulletin/volumes/89/12/BLT-11-091660-table-T3.html}{summary}}
relationship between the introduction of the Mexico City Policy and an
increase in induced abortions.

The U.S. Administration of International Development (USAID), as the
major government agency managing international aid, is responsible for
reporting all official U.S. foreign aid to Congress. USAID maintains an
extensively detailed database for reporting this data known as the
Greenbook, which keeps records of foreign aid obligations and
disbursements dating back to 1946. In addition to storing and reporting
this data, USAID provides numerous open data development projects in
order to encourage public collaboration on research and data analysis.
The Foreign Aid Explorer\footnote{USAID is responsible for reporting
  official U.S. Government foreign aid to Congress and the Organization
  for Economic Cooperation and Development (OECD)
  \href{https://explorer.usaid.gov/aid-dashboard.html}{Foreign Aid
  Explorer website}} provides a dashboard for querying and downloading
this dataset, but also includes its own visualizations and maps that
provide informative breakdowns of global financial aid. This project was
valuable for obtaining and becoming familiar with the USAID Greenbook
dataset. To provide additional context for our analysis we referenced
official statements from many of the NGOs that we identified as
prominent in the reproductive health and family planning sector that
clarify their activities in abortion counselling and their stances on
the Mexico City Policy.

\section{Methods}\label{methods}

While the World Contraceptive Use 2016 dataset included levels of
contraceptive and family planning methods for an extensive number of
countries across the world, we chose to analyze a certain subset of this
data. After the raw excel data was downloaded from the United Nations'
website, it required a substantial amount of cleaning because of the
format of the column headers. Once the columns were renamed, records for
countries in sub-Saharan Africa were aggregated over time using dplyr.
Reproductive health metrics for married/in-union women of reproductive
age were examined across the time periods where the Mexico City Policy
was active and inactive in the United States. Our hypothesis was that
during periods where the policy was actively blocking funding,
contraceptive prevalence would decrease, and unmet needs for family
planning would increase. This would make sense because the policy blocks
the entirety of funding for an organization if it provides any abortion
counseling, so many NGOs would not be able to operate.

In approaching this analysis we needed to use some unconventional
methods of data sourcing. For the data on induced abortions and exposure
to the ``Global Gag Rule'' (GGR) we scraped a \texttt{.html} table that
the WHO made available through the Stanford paper published on their
website. Unfortunately for us, the hierarchy of the table did not easily
lend itself to reshaping in R so a considerable amount of work was done
to convert it into a long form \texttt{.csv} file. We were curious about
some of the potential public health effects that the ban could have had
historically on an area such as maternal mortality. The hypothesis that
maternal mortality would be higher after the ban was instituted was
based on our intuition. We used a dataset from the World Health
Organization covering maternal mortality rates from 1990 to 2015 for
every country. From there we broke the data into subsets to focus on the
20 sub-Saharan countries that we had exposure data on and added those
rates onto our dataset. After we had a complete dataset we performed a
linear regression to gauge whether our hypothesis was correct. In the
model we used the variation in policy exposure compared against
variation in the rates of maternal mortality while subsetting for
abortion rates in the targeted countries.

While the Foreign Aid Explorer dashboard presents a lot of useful
visualizations, the underlying USAID dataset captures a lot more
specific detail for each funding activity that we wanted to leverage to
ask more specific questions. The approach that we took to investigate
patterns related to the impact of the Mexico City Policy on NGO funding
involved exporting a subset of this dataset for use with our own custom
visualizations. We chose to limit our data to the fiscal years
2001-2016, focusing on aid disbursements to NGOs with a designated
`sector' value of `Health and Population'. This allowed us to easily
work with and categorize funding that is the most likely to be impacted
by the reinstatement of the policy. There were a few major questions we
had about this data. Primarily, we wanted to understand the historical
trends of USAID funding for NGOs that provide the specific services
targeted by the Mexico City Policy. Our goal was to attempt to provide
both an estimate of how much funding is `at stake' and which NGOs in
particular might be the most vulnerable to the implementation of the
policy. While the USAID funding data does not provide enough information
to determine whether a particular NGO provides abortion related
counselling, education, or medical services, it allows us to target data
for funding of activities with a specific `purpose'. By looking at
funding activity with a stated purpose of `family planning' our
intuition was that we would be likely to find numerous disbursements to
NGOs that are actively pro-choice and provide many of the specific
services outlined in the policy. Our first visualization of NGO aid
disbursements provides a plot summarizing the year-by-year funding of
the top 10 recipients of US foreign aid for `family planning', along
with a comparison of each of these NGOs to the total amount of funding
for family planning. While investigating this data, we observed that
many of the NGOs which account for a majority of this type of aid are
also highly active in other sectors related to global health and
population health efforts. Our second visualization allows for
exploration of the related efforts of some of the most-funded NGOs for
family planning activities, and attempts to provide insight into the
extent to which the global gag rule may impact funding for programs that
do not fall directly under the umbrella of abortion counselling and
medical services.

\section{Results}\label{results}

An analysis of the United Nations' data revealed several key insights.
Throughout the period of 1984 to 2014, among all Sub-Saharan African
countries,\footnote{This group of Sub-Saharan includes 21 countries
  fully or partially located south of the Sahara.} Zimbabwe and Cabo
Verde had the highest prevalence of contraceptives, especially modern
ones, which include sterilization, intra-uterine device (IUD), implant,
injectables, pills, condoms, barrier methods, LAM, emergency
contraception and others. This indicates a strong presence of sexual
education in the population. Zimbabwe, which has a ``low'' exposure to
the Mexico City Policy, as found in the WHO analysis, appears to have a
contraceptive prevalence level that is largely unaffected by the
activity and inactivity of the policy, as there is an upward increase
for every year except 2011. In 2014, 66.9\% of heterosexual couples in
Zimbabwe used modern or traditional contraception. On the other hand,
one year after, only 5.7\% of the same grouping in Chad used any form of
modern or traditional contraception, which was extremely low. Mali, a
country that is known to have a ``high'' exposure to the policy, did not
rise and fall in contraceptive prevalence as we anticipated in
correlation with changes in the legislation. Instead, it exhibited a
slower but fairly steady upward trend, just as Zimbabwe did (Figure 1.).
The proportion of family planning demand satisfied by modern methods
grew every year, however, the percent with unmet needs for family
planning rose until around 2001 when it began to fall. As with these,
and most of the other countries, there are not many obvious trends
during the time periods when the Mexico City Policy was in effect.

\begin{figure}
\centering
\includegraphics{final-paper_files/figure-latex/unnamed-chunk-1-1.pdf}
\caption{Married / In-Union Women}
\end{figure}

In exploring the differences in the trends of induced abortion rates for
the two main subsets of our data, we looked at the countries that were
rated with a low exposure to the policy and countries rated with a high
exposure. Though we saw an increase in abortion rates in countries that
are considered highly exposed to the policy, the statistical
significance was negligible.(Figure 2.) The two curves were calculated
from ``observational data'' using a locally weighted smoothing (lowess)
method. The linear regression model that we performed used the variation
in policy exposure evaluated against variation in the rates of maternal
mortality to determine of a connection could be made between the
variables. The model that we performed did not show that there was a
meaningful relationship.

\begin{figure}
\centering
\includegraphics{final-paper_files/figure-latex/unnamed-chunk-2-1.pdf}
\caption{Abortion Rates With High Vs. Low Exposure}
\end{figure}

Visualizing the flow of US foreign aid to NGOs provides significant
insight that helps to contextualize the impact of the global gag rule.
Observing the historical trends of family planning funding disbursed to
NGOs showed a significant increase in government aid following the
repeal of the Mexico City Policy by President Obama in 2009. While we
can't make definite claims that this increase is directly related to the
policy change, it illustrates that there has been significant growth in
efforts to fund and implement family planning services since that time.

\begin{figure}
\centering
\includegraphics{final-paper_files/figure-latex/unnamed-chunk-3-1.pdf}
\caption{U.S. Aid to NGOs for Family Planning}
\end{figure}

By narrowing our visualization to only the top 10 recipients of funding
for family planning activity, we were not only able to present a clearer
picture of the major organizations potentially affected by the policy,
but were also able to observe that the majority of this type of aid
funding passes through only a few large organizations. In particular, we
noted that John Snow International (JSI), a public health research firm,
accepts a large amount of USAID funding for family planning. These
findings extend to our breakdown of each NGO's total funding by purpose,
which indicates that JSI has received billions of additional dollars of
funding for work in other fields including Malaria control, STD control
including HIV/AIDS, and a wide variety of basic health concerns. For
each of the NGO's we examined, we found a wide variety of projects in
health and population sectors. These findings underscore the notion that
the global gag rule has the potential to disrupt advancement of aid
funding for highly important work in improving population health on a
much broader scale than maternal health alone.

\section{Discussion}\label{discussion}

Analysis of the United Nations' data showed some interesting differences
between sub-Saharan African countries with respect to contraceptive
prevalence and needs for family planning. It is clear that countries
like Zimbabwe have a more progressive use of contraceptives and family
planning methods, and ones like Mali exhibit much less. However, our
research sparks the question, why exactly is this the case? Economic and
social changes in these countries must be largely driving these changes,
in conjunction with the Global Gag Rule. While our initial intent was to
analyze the pure effects of the policy, it is clear that this is
difficult as there are many complex factors at play. Each country is
often undergoing radical changes that affect women's reproductive health
and family life, only one of which is the mentioned Mexico City Policy.

While we did find a small correlation between sub-Saharan countries
where exposure to the policy was high, it was not substantial enough to
be conclusive. It should also be noted that the abortion rate estimates
are lower on average than those reported in other countries. These
abortion rates have been adjusted to correct for underreporting. We were
also unable to find a statistically significant relationship between the
variables of maternal mortality rates and the presence of the policy.
This does not mean that there is no correlation between these variables,
it simply means that the datasets available to us lacked sufficient
amount of features to control for. Because of this we can only conclude
that more research should be done in this area. With datasets that
contain a wider range of variables we could control for a host of
external factors that may be influencing the outcome of the data.

Our results suggest that there is a significant degree of complexity to
the flow of US aid funding through NGOs, and that a non-negligible
amount of established funding is potentially at risk. Family planning
efforts are often interwoven into other activities and projects which
NGOs implement and focus on, such as STD prevention and general
reproductive health. The results of our categorical breakdown of NGO
funding motivates more formal research into the estimated effects of the
global gag rule on NGOs in the population and health sector.

\section{Future Work}\label{future-work}

Going forward, it is clear that the Mexico City policy has the potential
for sweeping negative public health effects particularly for maternal
health. Throughout our research the common theme has been the need for
more data points and features. In order to truly understand the global
implications of the policy, additional research must be conducted.
Perhaps with more time, funding, and manpower a better analysis could be
produced and an insight into the true effects of the policy reached.
With respect to USAID funding, there are more relationships that are
worth exploring in the Greenbook data. With a stronger background in
finance and more data on the budgets of specific NGOs, it may be
worthwhile to investigate which NGOs are the most ``at risk'' in terms
of the proportion of their budget that is provided by US foreign aid.
This analysis could be extended further to understand which populations
and regions would face the most significant loss of services and
assistance from NGOs that provide family planning services and other
vital health resources.

\newpage
\singlespacing 
\end{document}
